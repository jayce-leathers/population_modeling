\documentclass{article}

\usepackage{fancyhdr}
\usepackage{extramarks}
\usepackage{amsmath}
\usepackage{amsthm}
\usepackage{amsfonts}
\usepackage{tikz}
\usepackage[plain]{algorithm}
\usepackage{algpseudocode}

\usetikzlibrary{automata,positioning}

%
% Basic Document Settings
%

\topmargin=-0.45in
\evensidemargin=0in
\oddsidemargin=0in
\textwidth=6.5in
\textheight=9.0in
\headsep=0.25in

\linespread{1.1}

\pagestyle{fancy}
\lhead{\hmwkAuthorName\ \hmwkClass\ (\hmwkClassInstructor\ \hmwkClassTime): \hmwkTitle}
\rhead{\firstxmark}
\lfoot{\lastxmark}
\cfoot{\thepage}

\renewcommand\headrulewidth{0.4pt}
\renewcommand\footrulewidth{0.4pt}

\setlength\parindent{0pt}


%
% Homework Details
%   - Title
%   - Due date
%   - Class
%   - Section/Time
%   - Instructor
%   - Author
%

\newcommand{\hmwkTitle}{Project\ \#2}
\newcommand{\hmwkDueDate}{Nov 3, 2015}
\newcommand{\hmwkClass}{Modeling, Simulation, and Analysis}
\newcommand{\hmwkClassTime}{Section 1}
\newcommand{\hmwkClassInstructor}{Professor Eric Aaron}
\newcommand{\hmwkAuthorName}{Jayce Rudig-Leathers}

%
% Title Page
%

\title{
    \vspace{2in}
    \textmd{\textbf{\hmwkClass:\ \hmwkTitle}}\\
    \normalsize\vspace{0.1in}\small{Due\ on\ \hmwkDueDate\ at 10:30am}\\
    \vspace{0.1in}\large{\textit{\hmwkClassInstructor\ \hmwkClassTime}}
    \vspace{3in}
}

\author{\textbf{\hmwkAuthorName}}
\date{}



\begin{document}

\maketitle

\pagebreak
\vspace{.25cm}
\section*{Problem 1}

  \textbf{a}\\
  We first develop the equation for a model of three species' populations: P the predator,
  Y the prey and H the humans. We assume that P only hunts Y and that H hunts Y and P
  equally.  Also let us assume that the population H is constant over our simulation.
  Based on the ideas presented in chapter 4.2 of our text, we arrive
  at the following equations.\\

  \[
  \frac{dP}{dt} = K_{PY}PY - K_{H}HP - K_{P}P
  \]\\

  This equation represents the rate in change of the predator population. $P,Y and H$
  are simply the populations of their respective species at time $t$. $K_{PY}PY$ represents
  P's births due to interactions with Y, $K_{PY}$ is the proportion of these interactions
  resulting in P births. Similarily $K_{HP}HP$ represents interactions with H resulting
  in P deaths where $K_{H}$ is the proportion of H interactions resulting in deaths. This
  variable is the same for both equations.
  Lastly, $K_{P}P$ is the natural deaths of P with $K_{P}$ is the fraction of deaths of P.
  This is actually a revised model. The original did not take into account deaths due to
  natural causes however due to a problem with the model in part e it was revisited and
  this was added.

  \[
  \frac{dY}{dt} = K_{Y}Y - K_{YP}YP - K_{H}HY
  \]\\

  This equation represents the rate in change of the prey population. $P,Y and H$
  are simply the populations of their respective species at time $t$. $K_{Y}Y$
  represents the births of Y, $K_{Y}$ is the fraction of births of Y. $K_{YP}YP$
  represents deaths due to interactions with P, $K_{YP}$ is the proportion of interactions
  resulting in Y deaths. Similarily, $K_{H}HY$ represents the deaths due to H interactions.
  As said before H is the deaths of both species due to H. \\

  \textbf{b}\\

  For either equation if the population modeled reaches 0 the growth function goes to
  and stays at 0.
  This makes sense as the population (Y or P respectively) is a term of each quantity
  in the equation i.e. $K_{Y}0 - K_{YP}0P - K_{H}0Y = 0$. If a population is 0 and
  it's growth rate is 0 it will remain at
  0. We also find equilibria solutions if the positive quantities of each equation equals the
  negative thus the growth rate goes to zero and populations remain constant all other things
  constant. We can see that

  \[
    K_{PY}PY  =  K_{H}HP + K_{P}P \to K_{PY}Y  =  K_{H}H + K_{P}\\
    \text{ and }\\
    K_{Y}Y = K_{YP}YP + K_{H}HY \to K_{Y} = K_{YP}P + K_{H}H\\
  \]
\pagebreak
  are both equilibria solutions as the growth rate is 0 in both of these cases.\\

  \textbf{c}\\
  We begin with the model without fishing. It's initial variables are as below.
  The result can be seen in fig 1c\_01 within my submitted code folder.\\

  \begin{verbatim}
    %simulation variables
    dt = .00001; %time step
    simulationLength = 12; %length of time to simulate. In months
    N = simulationLength/dt; %number of iterations

    %xAxis array of timeSteps
    times = zeros(1,N + 1);
    times(1) = 0;

    %Population  Inital values
    Y(1) = 100;
    P(1) = 15;

    %Proportion of Interaction Variables
    KPY = .01; %Proportion of Predator births due to interactions with Prey
    KYP = .02; %Proportion of Prey deaths due to interactions with Predators
    KY = 2; %Prey birth fraction
    KP = 1.06; %Predator death fraction
  \end{verbatim}

  We then set P(1) = 50  and  get more moderate oscillation (fig 1c\_02).
  Then try high P low (~1000) and Y (~5) and the reverse. The oscillations became
  more severe (sharp) in both cases (fig 1c\_03). We then try Y and P equal at 100.
  (fig 1c\_04) from this we can conclude that the difference between Y and P determine
  the amount the functions oscillate. The greater the distance the higher the
  fluctuations. This makes sense because if one of the populations is disproportionately
  large the other must increase  nearer that level before it can have a significant
  impact on the larger.\\

  Varying the KPY rate (.001,.1) reveals that a higher KPY raises predator population
  constantly above prey (fig 1c\_05), while a very low KPY causes the prey population
  to increase rapidly (fig 1c\_06). This makes sense if we think about KPY as how well
  P hunts Y. i.e. how many possible interactions result in a birth. So with high
  KPY there are more births and the predator population stays higher than prey.\\

  KYP affects the model in a similar way. We can think of KYP as the vulnerability
  of the prey’s defense. So an increased KYP (.1) results in a small population of
  predators being able to control a large population of prey (fig 1c\_07). Or conversely
  a very low vulnerability (.005) means that the predator population has to grow
  huge before it decreases the growth of prey population (fig 1c\_08).\\

  The KY and KP both act in similar ways. KY (.5) increases how quickly new prey
  are born and thus increases how sharply prey and predator populations grow (fig 1c\_09).
  And KP (4.06) increases how quickly predators die with similar affects on the model
  (fig 1c\_10).\\

  We then add a constant human population to the model as follows\\

  \begin{verbatim}
    %simulation variables
    dt = .00001; %time step
    simulationLength = 12; %length of time to simulate. In months
    N = simulationLength/dt; %number of iterations

    %xAxis array of timeSteps
    times = zeros(1,N + 1);
    times(1) = 0;
    %Population Initial values

    Y(1) = 100;
    P(1) = 15;
    H(1:N + 1) = 25;

    %Proportion of Interaction Variables
    KPY = .01; %Proportion of Predator births due to interactions with Prey
    KYP = .02; %Proportion of Prey deaths due to interactions with Predators
    KH = .01; %Proportion of deaths due to human interactions same for predators and prey
    KY = 2; %Prey birth fraction
    KP = 1.06; %Predator death fraction
  \end{verbatim}

  We don't initially see much difference (fig 1c\_11, 1c\_12) except for a slight lowering
  in the maximum population P attains. As we increase KH the difference between P and Y
  becomes even more pronounced as in fig 1c\_13 where KH is .06. Raising KH raises Y while
  decreasing P. I attribute this to the fact that H both hunts P and its food supply
  doubley decreasing the population. Y is raised because it's main predator P is shrinking.\\

  \textbf{d}\\
  We then increase KH to .07 and see the same effects as before. At .08, however,
  P goes extinct and Y stays constant (fig 1c\_14). This is because once P is driven extinct
  not only has it reached equilibrium but so has Y. If we plug our current variables into
   $K_{Y} = K_{YP}P + K_{H}H$ the equilibrium solution we calculated earlier
  we get $2 = (.02 * 0) + (.08 * 25) \to 2 = 2$ which is why the Y population stays constant.\\

  \textbf{e}\\
  This model takes into account a periodic rate of H hunting. It was built based on
  the formula for cosine given in chapter 4.2 of our text $f + a cos ( p * t)$,
  where p is the period, a the amplitude and f a vertical shift. I solve for a period of 12
  to get $p = frac{\pi}{6}$ amplitude is set to .02 to start (this is the max of KH rate)
  , and $f = 0$. We can see the KHT function in fig 1e\_01.
  I encountered many problems simulating this due to my previous exclusion of natural predator
  deaths in the model. This caused the system to act in unexpected and undesired ways.
  After I realized that predators should die naturally too this results are much more
  meaningful. The resulting model is the same as previously presented in parts e and d except
  that KH is a function of time. This yields the following equations\\

  \[
    \frac{dP}{dt} = K_{PY}PY - K_{H}(t)HP - K_{P}P
  \]
  \[
    \frac{dY}{dt} = K_{Y}Y - K_{YP}YP - K_{H}(t)HY
  \]

  \pagebreak
  Our simulations starts as follows\\

  \begin{verbatim}
    %Initial values
    Y(1) = 250;
    P(1) = 150;
    H(1:N + 1) = 10; %constant

    %Proportion of Interaction Variables
    KPY = .01; %Proportion of Predator births due to interactions with Prey
    KYP = .02; %Proportion of Prey deaths due to interactions with Predators
    KY = 2; %Prey birth fraction
    KP = 1.06; %Predator death fraction

    %cosine constants
    f = 0; %raise graph by f
    a = .04; %amplitude - this ends up being the max value of
    p = pi/6; %period
  \end{verbatim}
  This first simulation results in fig 1e\_02. We see the P population peak higher
  after the second KHT peak but not the first. The second simulation we increase the
  simulation length to 48 and the pattern is more visible (fig 1e\_03). Following alternating
  peaks of KHT the Y level will spike higher and then the P level will dip lower. If we
  extend the simulation length to 144 we get fig 1e\_05. At this point, simulation error
  begins to alter the graph but we can see that the pattern is actually more complicated.
  We can also see the fishing rate (green) plotted along with the populations. It appears
  that the fluctuating heights of the peaks are due to the difference between the period of
  the fishing rate and the natural cycles of P and Y interaction. In other words the height
  depends on where
  P and Y are in their cycle (as illustrated in simulation one) when the fishing begins.
  We then bring the simulation length down to 48 in order to only see relatively
  accurate results. In fig 1e\_06 we see that the highest prey peaks are after
  a full period of P and Y have happened while the fishing rate was zero
  (right around t = 8 and t = 30). It's after the points when the increasing fishing rate
  drives P to it's lowest trough that the Y rate spikes so sharpyly. I attribute this to
  the fact that there are less predators resulting in less prey deaths so Y can grow
  as sharply as we see. If we change the max fishing rate we see that the higher the rate
  the sharper and higher Y and P peak, (fig 1e\_07 with rate $\le .08$ and fig 1e\_08 with
  rate $\le .01$). Some other interesting results are that if we increase the human population to
  50 both P and Y will go extinct fig 1e\_09. Y goes extinct first and then without food
  P follows. If we then leave the human population at 50 and increase the Y birth rate, we
  get fig 1e\_10. Here we see that with more prey in the system it stabilizes. I believe that
  the fluctuations after t = 35 are due to simulation error. These results all model
  as it brings up many questions about the robustness of a system to withstand human interactions.

  \pagebreak
  \section*{Problem 2}

  \vspace{.5cm}
  \textbf{a}\\
  We develop the following equations for our new model. This is largely based on our
  previous model but with H not constant. We assume that H's only prey is P so it's births
  are represent by $K_{HP}HP$ where $K_{HP}$ is the proportion of P interactions resulting in
  births. and that
  it's deaths are only a function of the current population. The only change to Y's
  equation is that Y and H no longer directly interact. The equation for P has not changed
  except changing the proportion of H interactions resulting in death's notation to be $K_{PH}$ and
  removing naturally deaths based on the instructions to treat P's deaths as those of a prey species.

  \[
    \frac{dP}{dt} = K_{PY}PY - K_{PH}PH
  \]
  \[
    \frac{dY}{dt} = K_{Y}Y - K_{YP}YP
  \]
  \[
    \frac{dH}{dt} = K_{HP}HP - K_{H}H
  \]

  \textbf{b}\\

  From the model in part a, we then arrive at the following model. The only differences here are
  that first $\frac{dY}{dt}$ takes into account deaths due H interactions $K_{YH}YH$ where $K_{YH}$ is
  the proportions of interactions resulting in death. And $\frac{dH}{dt}$ takes into account
  births due to the same interactions where $K_{HY}$ represents this proportion.
  \[
    \frac{dP}{dt} = K_{PY}PY - K_{PH}PH
  \]
  \[
    \frac{dY}{dt} = K_{Y}Y - K_{YP}YP - K_{YH}YH
  \]
  \[
    \frac{dH}{dt} = K_{HP}HP + K_{HY}HY - K_{H}H
  \]

  \textbf{b}\\

  I wonder whether H hunting both Y and P or if H hunting only P will result in a more
  fragile system. By fragile I mean how easy it easy for one of the species to go extinct.

  My hypothesis is that in model b, P will be driven to extinction if KHP, KHY or KH is
  large compared to KPY
  While in model a, P will harder to drive extinct.
  I base this on my suspicion that H will have an inordinate impact on the system due to the fact
  that it hunts both species Y and P.

  We start with the simulation as seen here. Most constants are as they are in the model
  from 1e. The new constants in the model are best guesses based on other constants involved.\\

\begin{verbatim}
  %simulation variables
  dt = .000001; %time step
  simulationLength = 24; %length of time to simulate. In months
  N = simulationLength/dt; %number of iterations

  %Inital values
  Y(1) = 150;
  P(1) = 50;
  H(1:N + 1) = 25;

  %Proportion of Interaction Variables
  KPY = .01; %Proportion of Predator births due to interactions with Prey
  KPH = .01; %Proportion of Predator deaths due to H
  KP = 1.06; % P death fraction

  KYP = .02; %Proportion of Prey deaths due to interactions with Predators
  KYH = .01; %Proportion of Prey deaths due to interactions with H
  KY = 2; %Prey birth fraction

  KHP = .01; %Proportion of H births due to P interactions
  KHY = .01; %Proportion of H births due to Y interactions
  KH = 1.01; %Human death fraction an order of magnitude greater than constants
             %involving two populations
\end{verbatim}

This simulation results in fig 2c\_01 and fig 2c\_02. Fig 2c\_01 of model a
seems to show a functional population system. There are easy cycles and nothing goes
extinct. However,in fig 2c\_02 of model b, we see that P does go extinct. These results
don't necessarily support our hypothesis we can't be sure where exactly they come from
since\\

\[
  KHP = .01 = KPY
\]
\[
  KYH = .01 = KPY
\]
\[
  KH = 1.01 = KPY * 10 \text{(compare by order of magnitude)}
\]

We begin by increasing KPY by .01 to see how that affects the system. We can see
in fig 2c\_03 that model a has only changed by raising the maximum P and H, and lowering
the maximum of Y. In fig 2c\_4, we see that P no longer goes extinct in model b with this
change. This supports our hypothesis, but further testing is needed. Leaving KPY as is
and raising KPH to .02 (KPH = KPY again) we get no significant change in model a.
 In fig 2c\_05 we see that once
again P goes extinct. Let's now test when KPH is higher or lower than KPY. Raising KPH
to .03, .04 and .05 we see that the only change is how quickly P goes extinct (fig 2c\_06).
Not even at .05 does P go extinct in model a.
Lowering KPH  to .005 we see that the period of the graph is lengthened and that P and H
grow higher than with KPH = .02, but do not go extinct. These results do seem to support
our hypothesis with the added insight that P will go extinct if $KPH \ge KPY$.\\

Leaving KPH at .01 (the value at which P does not go extinct), we then explore the effects of
KYH (KPH is not in model a so it remains unchanged).  We raise KYH to .02 and get a lengthened periodicity but not extinction. Raising it all
the way to .1 does not result in an extinction of P, just a very stretched period (fig 2c\_07).
Increasing the KYH all the way to 1 does not in fact drive P to extinction but does drive
Y and H to extinction (fig 2c\_08). We can conclude from this that KPY and KYH are not
related as hypothesised. This result is directly because we chose not to factor natural
deaths into P. So when Y and H die out there are no P deaths and thus P remains constant
as we see.\\

We now return KPH to .01 and explore KH's effects. Raising KH to 1.5 we see similar results
to the simulation targeting KPH. H drives Y extinct and then it self goes extinct. P then
becomes constant (fig 2c\_09). Model a's graph is very similar.  This is again due to our simplifying assumption of P as prey as far
as deaths are modeled. From these last two results I conclude that maybe this was not
the right assumption for our model. Running one last test with natural deaths added to
the model we get fig 2c\_10 where H goes extinct but P and Y remain in a normal pattern.

From these results we can conlude that only the first part of hypothesis is supported by the
data. That is that KHP greater than KPY will cause extinction of P in model b but not
model a.

\pagebreak
\end{document}
